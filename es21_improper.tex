\documentclass[11pt]{article}
%%%%%%%%%%PACKAGES%%%%%%%%%%%%%%%%%%%%%%%%%%%%%%%%%%%
\usepackage{latexsym}
\usepackage{amssymb, amsmath, amsthm, amsfonts}
\usepackage{stmaryrd} %For \mapsfrom
%\usepackage[fleqn]{amsmath}  % fleqn option makes aligned equations flushed left!
%\usepackage[english]{babel}
%\usepackage{pgf}
\usepackage{mathtools}
\usepackage[mathscr]{eucal}
\usepackage{fancyhdr}
\usepackage{multicol,parcolumns}
\usepackage{enumerate}
%\usepackage{enumitem}
\usepackage[shortlabels]{enumitem}
\usepackage{graphicx}
\usepackage{extarrows}
\usepackage{cancel}
%\usepackage{tikz}
%\usepackage[all,cmtip]{xy} %\SelectTips{cm}{10}
\usepackage[all]{xy} \SelectTips{cm}{10}
%\usepackage{listings} %For code blocks

%%%%Blackboard Bold%%%%%
\newcommand{\N}{{\mathbb N}}
\newcommand{\Z}{{\mathbb Z}}
\newcommand{\Q}{{\mathbb Q}}
\newcommand{\R}{{\mathbb R}}
\newcommand{\C}{{\mathbb C}}
\newcommand{\T}{{\mathbb T}}
\newcommand{\F}{{\mathbb F}}
\newcommand{\HH}{{\mathbb H}}

\newcommand{\compose}{\circ}
%%%%%Bold%%%%%%%%%
\newcommand{\bolda}{{\mathbf a}}
\newcommand{\boldb}{{\mathbf b}}
\newcommand{\boldc}{{\mathbf c}}
\newcommand{\boldd}{{\mathbf d}}
\newcommand{\bolde}{{\mathbf e}}
\newcommand{\boldi}{{\mathbf i}}
\newcommand{\boldj}{{\mathbf j}}
\newcommand{\boldk}{{\mathbf k}}
\newcommand{\boldn}{{\mathbf n}}
\newcommand{\boldp}{{\mathbf p}}
\newcommand{\boldq}{{\mathbf q}}
\newcommand{\boldr}{{\mathbf r}}
\newcommand{\bolds}{{\mathbf s}}
\newcommand{\boldt}{{\mathbf t}}
\newcommand{\boldu}{{\mathbf u}}
\newcommand{\boldv}{{\mathbf v}}
\newcommand{\boldw}{{\mathbf w}}
\newcommand{\boldx}{{\mathbf x}}
\newcommand{\boldy}{{\mathbf y}}
\newcommand{\boldz}{{\mathbf z}}
\newcommand{\boldzero}{{\mathbf 0}}
\newcommand{\boldmod}{\boldsymbol{ \bmod }}

\newcommand{\boldC}{{\mathbf C}}
\newcommand{\boldD}{{\mathbf D}}
\newcommand{\boldT}{{\mathbf T}}
\newcommand{\boldN}{{\mathbf N}}
\newcommand{\boldB}{{\mathbf B}}
\newcommand{\boldF}{{\mathbf F}}
\newcommand{\boldS}{{\mathbf S}}
\newcommand{\boldG}{{\mathbf G}}
\newcommand{\boldK}{{\mathbf K}}
\newcommand{\boldL}{{\mathbf L}}
\newcommand{\boldX}{{\mathbf X}}
\newcommand{\boldY}{{\mathbf Y}}
\newcommand{\boldZ}{{\mathbf Z}}
\newcommand{\boldH}{{\mathbf H}}
\newcommand{\boldR}{{\mathbf R}}

%%%%%%%%%%%OPERATORS%%%%%%%%%%%%%%%%%%
\renewcommand{\Re}{\operatorname{Re}}
\renewcommand{\Im}{\operatorname{Im}}

\DeclareMathOperator{\lcm}{lcm}
\DeclareMathOperator{\Span}{span}
\DeclareMathOperator{\tr}{tr}
\DeclareMathOperator{\NS}{null}
\DeclareMathOperator{\RS}{row}
\DeclareMathOperator{\CS}{col}
\DeclareMathOperator{\im}{im}
\DeclareMathOperator{\range}{range}
\DeclareMathOperator{\rank}{rank}
\DeclareMathOperator{\nullity}{nullity}
\DeclareMathOperator{\sign}{sign}
\DeclareMathOperator{\Fix}{Fix}
\DeclareMathOperator{\Aff}{Aff}
\DeclareMathOperator{\Frac}{Frac}
\DeclareMathOperator{\Ann}{Ann}
\DeclareMathOperator{\Tor}{Tor}
\DeclareMathOperator{\id}{id}
\DeclareMathOperator{\mdeg}{mdeg}
\DeclareMathOperator{\Lt}{Lt}
\DeclareMathOperator{\Lc}{Lc}
\DeclareMathOperator{\disc}{disc}
\DeclareMathOperator{\Frob}{Frob}
\DeclareMathOperator{\adj}{adj}
%\DeclareMathOperator{\proj}{proj}
\DeclareMathOperator{\curl}{curl}
\DeclareMathOperator{\grad}{grad}
\DeclareMathOperator{\diver}{div}
\DeclareMathOperator{\flux}{flux}
\DeclareMathOperator{\cis}{cis}
\DeclareMathOperator{\Arg}{Arg}
\DeclareMathOperator{\Log}{Log}
\DeclareMathOperator{\Arcsin}{Arcsin}
\DeclareMathOperator{\Arccos}{Arccos}
\DeclareMathOperator{\Arctan}{Arctan}
\DeclareMathOperator{\Res}{Res}
\DeclareMathOperator{\Int}{Int}
\DeclareMathOperator{\Ext}{Ext}
\DeclareMathOperator{\Isom}{Isom}
\DeclareMathOperator{\Nm}{Nm}
\DeclareMathOperator{\irr}{irr}
\def\Gal{\operatorname{Gal}}
\def\ord{\operatorname{ord}}
\def \ML {\operatorname{M}}
\def \GL {\operatorname{GL}}
\def \PGL {\operatorname{PGL}}
\def \SL {\operatorname{SL}}
\def \PSL {\operatorname{PSL}}
\def \GSp {\operatorname{GSp}}
\def \PGSp {\operatorname{PGSp}}
\def \Sp {\operatorname{Sp}}
\def \PSp {\operatorname{PSp}}
\def\Aut{\operatorname{Aut}}
\def\Inn{\operatorname{Inn}}
\def\Hom{\operatorname{Hom}}
\def\End{\operatorname{End}}
\def\ch{\operatorname{char}}


%%%%%%%Shortcuts and new commands %%%%%%%%
\def\Zp{\Z/p\Z}
\def\Zm{\Z/m\Z}
\def\Zn{\Z/n\Z}
\def\Fp{\F_p}

\newcommand{\surjects}{\twoheadrightarrow}
\newcommand{\injects}{\hookrightarrow}
\newcommand{\bijects}{\leftrightarrow}
\newcommand{\isomto}{\overset{\sim}{\rightarrow}}
\newcommand{\floor}[1]{\lfloor#1\rfloor}
\newcommand{\ceiling}[1]{\left\lceil#1\right\rceil}
\newcommand{\mclass}[2][m]{[#2]_{#1}}
\newcommand{\val}[2][]{\left\lvert #2\right\rvert_{#1}}
\newcommand{\abs}[2][]{\left\lvert #2\right\rvert_{#1}}
\newcommand{\valuation}[2][]{\left\lvert #2\right\rvert_{#1}}
\newcommand{\norm}[1]{\left\lVert#1\right\rVert}
\newcommand{\anpoly}{a_nx^n+a_{n-1}x^{n-1}\cdots +a_1x+a_0}
\newcommand{\anmonic}{x^n+a_{n-1}x^{n-1}\cdots +a_1x+a_0}
\newcommand{\bmpoly}{b_mx^m+b_{m-1}x^{m-1}\cdots +b_1x+b_0}
\newcommand{\pder}[2]{\frac{\partial#1}{\partial#2}}
\renewcommand{\c}{\cancel}
\newcommand{\normalin}{\trianglelefteq}
\newcommand{\angvec}[1]{\langle #1\rangle}
\newcommand{\varpoly}[2]{#1_{#2}x^{#2}+#1_{#2-1}x^{#2-1}\cdots +#1_1x+#1_0}
\newcommand{\varpower}[1][a]{#1_0+#1_1x+#1_1x^2+\cdots}
\newcommand{\limasto}[2][x]{\lim_{#1\rightarrow #2}}
\newcommand{\proj}[2]{\mbox{proj}_{#2}({#1}) }
\newcommand{\notimplies}{
        \mathrel{{\ooalign{\hidewidth$\not\phantom{=}$\hidewidth\cr$\implies$}}}}
\def\ntoinfty{\lim_{n\rightarrow\infty}}
\def\xtoinfty{\lim_{x\rightarrow\infty}}

\def\ii{\item}
\def\bb{\begin{enumerate}}
\def\ee{\end{enumerate}}
\def\ds{\displaystyle}
\def\p{\partial}

%\newenvironment{linsys}[2][m]{%
%\setlength{\arraycolsep}{.1111em} % p. 170 TeXbook; a medmuskip
%\begin{array}[#1]{@{}*{#2}{rc}r@{}}
%}{%
%\end{array}}

%\newenvironment{solution}{\begin{proof}[Solution]}{\end{proof}}

%%%%%%%%FANCY HEADER%%%%%%%%%
\pagestyle{plain}
\setlength{\headheight}{13.6pt}
\fancyhfoffset[L]{.5in}
%\lhead{\Large \bf{Name:}}
\chead{Executive summary: improper integrals}
\rhead{Math 220-2}
%\lfoot{TURN OVER!}
%\rfoot{TURN OVER!}

%%%%%%%PAGE LAYOUT%%%%%%%%%%%%%
\setlength{\textwidth}{6.5in}
\setlength{\textheight}{9in}

%\setlength{\topmargin}{-.8in}
%\setlength{\columnsep}{1.5in}
\addtolength{\hoffset}{-1 in}
\addtolength{\voffset}{-.5 in}


%%%%%%%THEOREM ENVIRONMENTS%%%%%%%%
\theoremstyle{definition}
\newtheorem*{definition}{Definition}
\newtheorem*{definitions}{Definitions}
\newtheorem*{notation}{Notation}
\newtheorem*{example}{Example}
\newtheorem*{comment}{Comment}
\newtheorem*{comments}{Comments}
\newtheorem*{examples}{Examples}
\newtheorem*{warning}{Warning}
\newtheorem*{theorem}{Theorem}
\newtheorem*{corollary}{Corollary}
\newtheorem*{proposition}{Proposition}
\newtheorem*{lemma}{Lemma}

\newtheoremstyle{named}{}{}{}{}{\bfseries}{.}{.5em}{\thmnote{#3}}
\theoremstyle{named}
\newtheorem*{namedtheorem}{Theorem}

\newcounter{myalgctr}
\newenvironment{myalg}{%      define a custom environment
   \bigskip\noindent%         create a vertical offset to previous material
   \refstepcounter{myalgctr}% increment the environment's counter
   \textbf{Algorithm \themyalgctr}% or \textbf, \textit, ...
   \newline%
   }{\par\bigskip}  %
\numberwithin{myalgctr}{section}



\newenvironment{solution}{\begin{proof}[Solution]}{\end{proof}}


%%%%%%%%%%HYPERREFS PACKAGE%%%%%%%%%%%%%%%%%
\usepackage[colorlinks]{hyperref}
%\definecolor{webcolor}{rgb}{0.8,0,0.2}
%\definecolor{webbrown}{rgb}{.6,0,0}
%\usepackage[
%        colorlinks,
%       linkcolor=webbrown,  filecolor=webcolor,  citecolor=webbrown,
%        backref
%]{hyperref}
\usepackage[alphabetic, lite]{amsrefs} % for bibliography
\begin{document}
\thispagestyle{fancy}
\subsection*{Definitions}
\begin{namedtheorem}[Improper integral of type I: infinite intervals] Below we define definite integrals over infinite intervals. These are called {\bf improper integrals of type I}, or {\bf integrals over infinite intervals}.
\begin{description}[topsep=0pt, itemsep=0pt]
  \item[Half-infinite intervals]\ \\ Definite integrals over intervals of the form $[a,\infty)$ or $(-\infty, a]$ are defined via the limit expressions below. When the relevant limit exists, we say the improper integral {\bf converges} (or {\bf exists}); otherwise we say the improper integral {\bf diverges}.
  \begin{itemize}[topsep=0pt, itemsep=0pt]
    \item Let $f$ be continuous on the interval $I=[a,\infty)$. We define the integral of $f$ over $I$, denoted $\displaystyle\int_a^\infty f(x)\, dx$, as the following limit, assuming it exists:
    \[
    \int_a^\infty f(x)\, dx=\lim_{R\to\infty}\int_a^R f(x)\, dx.
    \]
    \item Let $f$ be continuous on the interval $I=(-\infty,a]$. We define the integral of $f$ over $I$, denoted $\displaystyle\int_{-\infty}^a f(x)\, dx$, as the following limit, assuming it exists:
    \[
    \int_{-\infty}^af(x)\, dx=\lim_{R\to-\infty}\int_R^{a} f(x)\, dx.
    \]
  \end{itemize}
  \item[Real line]\ \\ Let $f$ be continuous on the interval $I=(-\infty,\infty)$, and let $a$ be an element of $I$. We say the integral of $f$ over $I$ {\bf converges} (or {\bf exists}) if {\em both} of the half-infinite integrals $\displaystyle\int_{-\infty}^a f(x)\, dx$ and $\displaystyle\int_a^{\infty}f(x)\, dx$ converge, and define
  \[
  \int_{-\infty}^\infty f(x)\, dx=\int_{-\infty}^a f(x)\, dx+\int_a^{\infty}f(x)\, dx
  \]
  in this case. If {\em either} (or both) of the half-infinite integrals diverge, we say that the integral of $f$ over $(-\infty, \infty)$ {\bf diverges}.
\end{description}
\end{namedtheorem}
\begin{samepage}
\begin{namedtheorem}[Improper integrals of type II: discontinuities] Assume $f$ is continuous on the interval $I=[a,b]$, except possibly at one point.
  \begin{itemize}
    \item Assume $f$ is not continuous at $x=a$. We define the integral of $f$ over $[a,b]$ as
    \[
    \int_a^bf(x)\, dx=\lim_{c\to a^+}\int_c^b f(x)\, dx,
    \]
    assuming this limit exists.
    \item Assume $f$ is not continuous at $x=b$. We define the integral of $f$ over $[a,b]$ as
    \[
    \int_a^bf(x)\, dx=\lim_{c\to b^-}\int_a^c f(x)\, dx,
    \]
    assuming this limit exists.
    \item Assume $f$ is not continuous at $c\in (a,b)$. We define the integral of $f$ over $[a,b]$ as
    \[
    \int_a^bf(x)\, dx=\int_a^c f(x)\, dx+ \int_c^b f(x)\, dx, \tag{$*$}
    \]
    assuming both improper integrals on the right side of $(*)$ exist.
  \end{itemize}

\end{namedtheorem}
\end{samepage}
\begin{namedtheorem}[Area interpretation of improper integrals] Let $f$ be defined on an interval $I$ for which the corresponding integral is improper, and let $\mathcal{R}$ be the (potentially unbounded) region between the graph of $f$ and the $x$-axis over the interval $I$.
  \begin{itemize}
  \item We define the {\bf area} (or {\bf total area}) of $\mathcal{R}$ to be the integral of $\lvert f\rvert$ over $I$, assuming this integral converges.
  \item We define the {\bf signed area} of $\mathcal{R}$ to be the integral of $f$ over $I$, assuming this interval converges.
\end{itemize}
\end{namedtheorem}

%***********************************************
 \subsection*{Theory}
\begin{namedtheorem}[Direct comparison test] Let $f$ and $g$ be {\em nonnegative} functions on an interval $I$, and suppose $f(x)\leq g(x)$ for all $x$ in $I$. If the integral of $g$ over $I$ converges, then the integral of $f$ over $I$ converges. Using logical notation:
  \[
  \text{integral of $g$ over $I$ converges }\implies \text{ integral of $f$ over $I$ converges}.
  \]
  Equivalently,
  \[
  \text{integral of $f$ over $I$ diverges }\implies \text{ integral of $g$ over $I$ diverges}.
  \]
\end{namedtheorem}

\begin{namedtheorem}[Limit comparison test] Let $f$ and $g$ be continuous and {\em positive} on the interval $I$.
  \begin{itemize}
    \item If $I=[a,\infty)$ and $\displaystyle\lim_{x\to\infty}\frac{f(x)}{g(x)}=L$ with $0< L <\infty$, then
    \[
    \int_a^\infty f(x)\, dx \text{ converges }\iff \int_a^\infty g(x)\, dx \text{ converges}.
    \]
    \item If $I=(-\infty,a]$ and $\displaystyle\lim_{x\to-\infty}\frac{f(x)}{g(x)}=L$ with $0< L <\infty$, then
    \[
    \int_{-\infty}^a f(x)\, dx \text{ converges }\iff \int_{-\infty}^a g(x)\, dx \text{ converges}.
    \]
    \item If $I=(a,b]$ and $\displaystyle\lim_{x\to a^+}\frac{f(x)}{g(x)}=L$ with $0< L <\infty$, then
    \[
    \int_{a}^b f(x)\, dx \text{ converges }\iff \int_a^b g(x)\, dx \text{ converges}.
    \]
    \item If $I=[a,b)$ and $\displaystyle\lim_{x\to b^-}\frac{f(x)}{g(x)}=L$ with $0< L <\infty$, then
    \[
    \int_{a}^b f(x)\, dx \text{ converges }\iff \int_a^b g(x)\, dx \text{ converges}.
    \]
  \end{itemize}
\end{namedtheorem}


%***************************************

%\subsection*{Procedures}

%*********************************************************
\subsection*{Examples}
\begin{enumerate}
  \item Evaluate $\displaystyle\int_{-2}^{\infty}e^{-x}\, dx$.
  \item Evaluate $\displaystyle\int_{0}^\infty xe^{-x}\, dx$.
  \item Evaluate $\displaystyle\int_{1}^\infty x^{r}\, dx$ for $r\ne 0$.
  \item Evaluate $\displaystyle\int_{-\infty}^{\infty}\frac{1}{x^2+1}\, dx$.
  \item Decide whether $\displaystyle\int_2^\infty \frac{1}{x^5+\sqrt{x+3}}\, dx$ converges.
  \item Decide whether $\displaystyle\int_1^\infty \frac{2+\sin x}{x}\, dx$ converges.
  \item Let $f(x)=ax^2+bx+c$ be any fixed irreducible quadratic polynomial with $a>0$. Decide whether $\displaystyle\int_{-\infty}^\infty \frac{1}{f(x)}\, dx$ exists.
  \item Evaluate $\displaystyle\int_{0}^2\frac{1}{x-1}\, dx$.
  \item Evaluate $\displaystyle\int_0^{1}\ln x\, dx$.
  \item Evaluate $\displaystyle\int_1^{4}\frac{x}{\sqrt[3]{x^2-4}}\, dx$.
  \item Decide whether $\displaystyle\int_0^\infty\frac{1}{\sqrt{x}+3x^5}\, dx$ converges.
\end{enumerate}




\end{document}
