\documentclass[11pt]{article}
%%%%%%%%%%PACKAGES%%%%%%%%%%%%%%%%%%%%%%%%%%%%%%%%%%%
\usepackage{latexsym}
\usepackage{amssymb, amsmath, amsthm, amsfonts}
\usepackage{stmaryrd} %For \mapsfrom
%\usepackage[fleqn]{amsmath}  % fleqn option makes aligned equations flushed left!
%\usepackage[english]{babel}
%\usepackage{pgf}
\usepackage{mathtools}
\usepackage[mathscr]{eucal}
\usepackage{fancyhdr}
\usepackage{multicol,parcolumns}
\usepackage{enumerate}
%\usepackage{enumitem}
\usepackage[shortlabels]{enumitem}
\usepackage{graphicx}
\usepackage{extarrows}
\usepackage{cancel}
%\usepackage{tikz}
%\usepackage[all,cmtip]{xy} %\SelectTips{cm}{10}
\usepackage[all]{xy} \SelectTips{cm}{10}
%\usepackage{listings} %For code blocks

%%%%Blackboard Bold%%%%%
\newcommand{\N}{{\mathbb N}}
\newcommand{\Z}{{\mathbb Z}}
\newcommand{\Q}{{\mathbb Q}}
\newcommand{\R}{{\mathbb R}}
\newcommand{\C}{{\mathbb C}}
\newcommand{\T}{{\mathbb T}}
\newcommand{\F}{{\mathbb F}}
\newcommand{\HH}{{\mathbb H}}

\newcommand{\compose}{\circ}
%%%%%Bold%%%%%%%%%
\newcommand{\bolda}{{\mathbf a}}
\newcommand{\boldb}{{\mathbf b}}
\newcommand{\boldc}{{\mathbf c}}
\newcommand{\boldd}{{\mathbf d}}
\newcommand{\bolde}{{\mathbf e}}
\newcommand{\boldi}{{\mathbf i}}
\newcommand{\boldj}{{\mathbf j}}
\newcommand{\boldk}{{\mathbf k}}
\newcommand{\boldn}{{\mathbf n}}
\newcommand{\boldp}{{\mathbf p}}
\newcommand{\boldq}{{\mathbf q}}
\newcommand{\boldr}{{\mathbf r}}
\newcommand{\bolds}{{\mathbf s}}
\newcommand{\boldt}{{\mathbf t}}
\newcommand{\boldu}{{\mathbf u}}
\newcommand{\boldv}{{\mathbf v}}
\newcommand{\boldw}{{\mathbf w}}
\newcommand{\boldx}{{\mathbf x}}
\newcommand{\boldy}{{\mathbf y}}
\newcommand{\boldz}{{\mathbf z}}
\newcommand{\boldzero}{{\mathbf 0}}
\newcommand{\boldmod}{\boldsymbol{ \bmod }}

\newcommand{\boldC}{{\mathbf C}}
\newcommand{\boldD}{{\mathbf D}}
\newcommand{\boldT}{{\mathbf T}}
\newcommand{\boldN}{{\mathbf N}}
\newcommand{\boldB}{{\mathbf B}}
\newcommand{\boldF}{{\mathbf F}}
\newcommand{\boldS}{{\mathbf S}}
\newcommand{\boldG}{{\mathbf G}}
\newcommand{\boldK}{{\mathbf K}}
\newcommand{\boldL}{{\mathbf L}}
\newcommand{\boldX}{{\mathbf X}}
\newcommand{\boldY}{{\mathbf Y}}
\newcommand{\boldZ}{{\mathbf Z}}
\newcommand{\boldH}{{\mathbf H}}
\newcommand{\boldR}{{\mathbf R}}

%%%%%%%%%%%OPERATORS%%%%%%%%%%%%%%%%%%
\renewcommand{\Re}{\operatorname{Re}}
\renewcommand{\Im}{\operatorname{Im}}

\DeclareMathOperator{\lcm}{lcm}
\DeclareMathOperator{\Span}{span}
\DeclareMathOperator{\tr}{tr}
\DeclareMathOperator{\NS}{null}
\DeclareMathOperator{\RS}{row}
\DeclareMathOperator{\CS}{col}
\DeclareMathOperator{\im}{im}
\DeclareMathOperator{\range}{range}
\DeclareMathOperator{\rank}{rank}
\DeclareMathOperator{\nullity}{nullity}
\DeclareMathOperator{\sign}{sign}
\DeclareMathOperator{\Fix}{Fix}
\DeclareMathOperator{\Aff}{Aff}
\DeclareMathOperator{\Frac}{Frac}
\DeclareMathOperator{\Ann}{Ann}
\DeclareMathOperator{\Tor}{Tor}
\DeclareMathOperator{\id}{id}
\DeclareMathOperator{\mdeg}{mdeg}
\DeclareMathOperator{\Lt}{Lt}
\DeclareMathOperator{\Lc}{Lc}
\DeclareMathOperator{\disc}{disc}
\DeclareMathOperator{\Frob}{Frob}
\DeclareMathOperator{\adj}{adj}
%\DeclareMathOperator{\proj}{proj}
\DeclareMathOperator{\curl}{curl}
\DeclareMathOperator{\grad}{grad}
\DeclareMathOperator{\diver}{div}
\DeclareMathOperator{\flux}{flux}
\DeclareMathOperator{\cis}{cis}
\DeclareMathOperator{\Arg}{Arg}
\DeclareMathOperator{\Log}{Log}
\DeclareMathOperator{\Arcsin}{Arcsin}
\DeclareMathOperator{\Arccos}{Arccos}
\DeclareMathOperator{\Arctan}{Arctan}
\DeclareMathOperator{\Res}{Res}
\DeclareMathOperator{\Int}{Int}
\DeclareMathOperator{\Ext}{Ext}
\DeclareMathOperator{\Isom}{Isom}
\DeclareMathOperator{\Nm}{Nm}
\DeclareMathOperator{\irr}{irr}
\def\Gal{\operatorname{Gal}}
\def\ord{\operatorname{ord}}
\def \ML {\operatorname{M}}
\def \GL {\operatorname{GL}}
\def \PGL {\operatorname{PGL}}
\def \SL {\operatorname{SL}}
\def \PSL {\operatorname{PSL}}
\def \GSp {\operatorname{GSp}}
\def \PGSp {\operatorname{PGSp}}
\def \Sp {\operatorname{Sp}}
\def \PSp {\operatorname{PSp}}
\def\Aut{\operatorname{Aut}}
\def\Inn{\operatorname{Inn}}
\def\Hom{\operatorname{Hom}}
\def\End{\operatorname{End}}
\def\ch{\operatorname{char}}


%%%%%%%Shortcuts and new commands %%%%%%%%
\def\Zp{\Z/p\Z}
\def\Zm{\Z/m\Z}
\def\Zn{\Z/n\Z}
\def\Fp{\F_p}

\newcommand{\surjects}{\twoheadrightarrow}
\newcommand{\injects}{\hookrightarrow}
\newcommand{\bijects}{\leftrightarrow}
\newcommand{\isomto}{\overset{\sim}{\rightarrow}}
\newcommand{\floor}[1]{\lfloor#1\rfloor}
\newcommand{\ceiling}[1]{\left\lceil#1\right\rceil}
\newcommand{\mclass}[2][m]{[#2]_{#1}}
\newcommand{\val}[2][]{\left\lvert #2\right\rvert_{#1}}
\newcommand{\abs}[2][]{\left\lvert #2\right\rvert_{#1}}
\newcommand{\valuation}[2][]{\left\lvert #2\right\rvert_{#1}}
\newcommand{\norm}[1]{\left\lVert#1\right\rVert}
\newcommand{\anpoly}{a_nx^n+a_{n-1}x^{n-1}\cdots +a_1x+a_0}
\newcommand{\anmonic}{x^n+a_{n-1}x^{n-1}\cdots +a_1x+a_0}
\newcommand{\bmpoly}{b_mx^m+b_{m-1}x^{m-1}\cdots +b_1x+b_0}
\newcommand{\pder}[2]{\frac{\partial#1}{\partial#2}}
\renewcommand{\c}{\cancel}
\newcommand{\normalin}{\trianglelefteq}
\newcommand{\angvec}[1]{\langle #1\rangle}
\newcommand{\varpoly}[2]{#1_{#2}x^{#2}+#1_{#2-1}x^{#2-1}\cdots +#1_1x+#1_0}
\newcommand{\varpower}[1][a]{#1_0+#1_1x+#1_1x^2+\cdots}
\newcommand{\limasto}[2][x]{\lim_{#1\rightarrow #2}}
\newcommand{\proj}[2]{\mbox{proj}_{#2}({#1}) }
\newcommand{\notimplies}{
        \mathrel{{\ooalign{\hidewidth$\not\phantom{=}$\hidewidth\cr$\implies$}}}}
\def\ntoinfty{\lim_{n\rightarrow\infty}}
\def\xtoinfty{\lim_{x\rightarrow\infty}}

\def\ii{\item}
\def\bb{\begin{enumerate}}
\def\ee{\end{enumerate}}
\def\ds{\displaystyle}
\def\p{\partial}

%\newenvironment{linsys}[2][m]{%
%\setlength{\arraycolsep}{.1111em} % p. 170 TeXbook; a medmuskip
%\begin{array}[#1]{@{}*{#2}{rc}r@{}}
%}{%
%\end{array}}

%\newenvironment{solution}{\begin{proof}[Solution]}{\end{proof}}

%%%%%%%%FANCY HEADER%%%%%%%%%
\pagestyle{plain}
\setlength{\headheight}{13.6pt}
\fancyhfoffset[L]{.5in}
%\lhead{\Large \bf{Name:}}
\chead{Executive summary: definite integral}
\rhead{Math 220-2}
%\lfoot{TURN OVER!}
%\rfoot{TURN OVER!}

%%%%%%%PAGE LAYOUT%%%%%%%%%%%%%
\setlength{\textwidth}{6.5in}
\setlength{\textheight}{9in}

%\setlength{\topmargin}{-.8in}
%\setlength{\columnsep}{1.5in}
\addtolength{\hoffset}{-1 in}
\addtolength{\voffset}{-.5 in}


%%%%%%%THEOREM ENVIRONMENTS%%%%%%%%
\theoremstyle{definition}
\newtheorem*{definition}{Definition}
\newtheorem*{definitions}{Definitions}
\newtheorem*{notation}{Notation}
\newtheorem*{example}{Example}
\newtheorem*{comment}{Comment}
\newtheorem*{comments}{Comments}
\newtheorem*{examples}{Examples}
\newtheorem*{warning}{Warning}
\newtheorem*{theorem}{Theorem}
\newtheorem*{corollary}{Corollary}
\newtheorem*{proposition}{Proposition}
\newtheorem*{lemma}{Lemma}

\newtheoremstyle{named}{}{}{}{}{\bfseries}{.}{.5em}{\thmnote{#3}}
\theoremstyle{named}
\newtheorem*{namedtheorem}{Theorem}

\newcounter{myalgctr}
\newenvironment{myalg}{%      define a custom environment
   \bigskip\noindent%         create a vertical offset to previous material
   \refstepcounter{myalgctr}% increment the environment's counter
   \textbf{Algorithm \themyalgctr}% or \textbf, \textit, ...
   \newline%
   }{\par\bigskip}  %
\numberwithin{myalgctr}{section}



\newenvironment{solution}{\begin{proof}[Solution]}{\end{proof}}


%%%%%%%%%%HYPERREFS PACKAGE%%%%%%%%%%%%%%%%%
\usepackage[colorlinks]{hyperref}
%\definecolor{webcolor}{rgb}{0.8,0,0.2}
%\definecolor{webbrown}{rgb}{.6,0,0}
%\usepackage[
%        colorlinks,
%       linkcolor=webbrown,  filecolor=webcolor,  citecolor=webbrown,
%        backref
%]{hyperref}
\usepackage[alphabetic, lite]{amsrefs} % for bibliography
\begin{document}
\thispagestyle{fancy}
\subsection*{Definitions}
\begin{namedtheorem}[Integrable function] Let $f$ be a function defined on the interval $[a,b]$. \\
  We say the {\bf definite integral of $f$ over $[a,b]$ exists} if there is a real number $J$ such that for {\em any} sequence of partitions $P_n$ of $[a,b]$ and {\em any} choice of Riemann sums $S_n$ corresponding to the partitions $P_n$, if the maximum width of a subinterval in $P_n$ approaches 0, then
  \[
  \lim_{n\to\infty} S_n=J.
  \]
In plain English, the definite integral of $f$ exists if any sequence of Riemann sums corresponding to a finer and finer partition of $[a,b]$ approaches the same value $J$ in the limit.

In this case we say {\bf $f$ is integrable over $[a,b]$} and call $J$ the {\bf definite integral of $f$ over $[a,b]$}, denoted
\[
\int_a^bf(x)\, dx=J.
\]
\end{namedtheorem}
\begin{namedtheorem}[Area and signed area of regions defined by functions] Let $f$ be integrable over the interval $[a,b]$, let $\mathcal{C}$ be the graph of $f$, and let $\mathcal{R}$ be the region between $\mathcal{C}$ and the $x$-axis from $x=a$ to $x=b$.
\begin{itemize}
  \item We define the {\bf area} (or {\bf total area}) of $\mathcal{R}$ to be the integral of $\lvert f\rvert$ over $[a,b]$: i.e.,
  \[
  \text{area of $\mathcal{R}$}=\int_a^b\lvert f(x)\rvert\, dx.
  \]
  \item We define the {\bf signed area} of $\mathcal{R}$ to be the integral of $f$ over $[a,b]$: i.e.,
  \[
  \text{signed area of $\mathcal{R}$}=\int_a^b f(x)\, dx.
  \]
\end{itemize}


\end{namedtheorem}
\begin{comment}
  Let $f$ be integrable over the interval $[a,b]$, let $\mathcal{C}$ be the graph of $f$, and let $\mathcal{R}$ be the region between $\mathcal{C}$ and the $x$-axis from $x=a$ to $x=b$.
\begin{enumerate}
  \item The area of $\mathcal{R}$ is always nonnegative, since $\lvert f(x)\rvert\geq 0$ for all $x\in [a,b]$.
  \item If $f(x)\geq 0$ for all $x\in [a,b]$, then $f=\lvert f\rvert$ over $[a,b]$, and hence
  \[
  \text{area of $\mathcal{R}$}=\int_a^b f(x)\, dx
  \]
  in this case.
  \item Suppose $[a,b]$ can be partitioned into finitely many intervals over which $f$ is either always nonnegative ($\geq 0$) or always nonpositive ($\leq 0$). Then
  \[
  \text{signed area of $\mathcal{R}$}=\int_a^b f(x)\, dx=(\text{area of regions where $f\geq 0$})-(\text{area of regions where $f\leq 0$}).
  \]
\end{enumerate}
\end{comment}

%***************************************

\subsection*{Procedures}
\begin{namedtheorem}[Direct computation of definite integral] Suppose $f$ is integrable on the interval $[a,b]$. \\
  Since $\displaystyle\int_a^bf(x)\, dx$ can be computed using any sequence of Riemann sums, we may compute it as a limit of right Riemann sums $R_n$ corresponding to partitions of $[a,b]$ into $n$ equal subintervals. For such partitions the length of each subinterval is $\Delta x=(b-a)/n$, and the right endpoint of the $k$-th subinterval is $x_k=a+k(b-a)/n$. We conclude:
  \[
  \int_a^b f(x)\, dx=\lim_{n\to\infty} R_n=\lim_{n\to\infty}\sum_{k=1}^n f\left(a+\frac{k(b-a)}{n}\right)\frac{(b-a)}{n}.
  \]

\end{namedtheorem}

%***********************************************
 \subsection*{Theory}
\begin{namedtheorem}[Integrable functions theorem] Let $f$ be defined on the interval $[a,b]$.
If $f$ is continuous everywhere on $[a,b]$, or if $f$ has at most finitely many jump discontinuities on $[a,b]$, then $f$ is integrable over $[a,b]$.
\end{namedtheorem}

\begin{namedtheorem}[Properties of definite integrals]
  Let $f$ and $g$ be integrable over $[a,b]$.
  \begin{enumerate}
    \item $\ds\int_a^a f(x)\, dx=0$. (By definition)
    \item $\ds\int_b^af(x)\, dx=-\int_a^b f(x)\, dx$. (By definition)
    \item {\bf Sum and difference}. $\ds\int_a^b f(x)\pm g(x)\, dx=\int_a^bf(x)\, dx\pm \int_a^bg(x)\, dx$
    \item {\bf Constant multiple}. $\ds\int_a^bcf(x)\, dx=c\int_a^bf(x)\, dx$ for any $c\in \R$.
    \item {\bf Additive}. For any $c\in \R$ we have
    \[
    \int_a^bf(x)\, dx=\int_a^cf(x)\, dx+\int_c^b f(x)\, dx,
    \]
    as long as all of the integrals involved are defined.
    \item {\bf Max-min inequality}. If $f$ has a minumum value $\min f$ on $[a,b]$ and a maximum value $\max f$ on $[a,b]$, then
    \[
    (\min f)(b-a)\leq \int_a^b f(x)\, dx\leq (\max f)(b-a)
    \]
    \item {\bf Domination}. If $f(x)\leq g(x)$ for all $x\in [a,b]$, then
    \[
    \int_a^bf(x)\, dx\leq \int_a^bg(x)\, dx.
    \]

  \end{enumerate}

\end{namedtheorem}

%*********************************************************
\subsection*{Examples}
\begin{enumerate}
    \item Fix positive constants $m$ and $b$, and define $f(x)=mx+b$.
    \begin{enumerate}
      \item Fix a positive constant $a$. Compute $\int_0^a f(x)\, dx$ directly as a limit of right Riemann sums.
      \item Graph $f(x)$ on $[0,a]$ and explain how your answer in (a) is consistent with known area formulas.
    \end{enumerate}
    \item Fix a positive constant $b$. Compute $\ds\int_0^b f(x)\, dx$ directly as a limit of right Riemann sums.

  \item Let $f(x)=1-x^3$. Fix constants $a$ and $b$ with $0<a<b$. Use your result in Example 2 and various integral properties (including the  additivite property) to derive a formula for $\ds\int_a^bf(x)\, dx$ in terms of $a$ and $b$.

  \item Let $f(x)=1-x^3$. Fix a constant $b$ with $b>1$, let $f(x)=1-x^3$, and let $\mathcal{R}$ be the region between the graph of $f$ and the $x$-axis from $x=0$ to $x=b$.
  \begin{enumerate}
    \item Graph $f(x)$ on $[0,b]$. Your graph should reflect the assumption that $b>1$.
    \item Describe precisely how the signed area of $\mathcal{R}$ is a difference of areas of two distinct regions.
    \item Compute the area of $\mathcal{R}$.
  \end{enumerate}
\end{enumerate}




\end{document}
