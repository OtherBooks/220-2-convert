\documentclass[11pt]{article}
%%%%%%%%%%PACKAGES%%%%%%%%%%%%%%%%%%%%%%%%%%%%%%%%%%%
\usepackage{latexsym}
\usepackage{amssymb, amsmath, amsthm, amsfonts}
\usepackage{stmaryrd} %For \mapsfrom
%\usepackage[fleqn]{amsmath}  % fleqn option makes aligned equations flushed left!
%\usepackage[english]{babel}
%\usepackage{pgf}
\usepackage{mathtools}
\usepackage[mathscr]{eucal}
\usepackage{fancyhdr}
\usepackage{multicol,parcolumns}
\usepackage{enumerate}
%\usepackage{enumitem}
\usepackage[shortlabels]{enumitem}
\usepackage{graphicx}
\usepackage{extarrows}
\usepackage{cancel}
%\usepackage{tikz}
%\usepackage[all,cmtip]{xy} %\SelectTips{cm}{10}
\usepackage[all]{xy} \SelectTips{cm}{10}
%\usepackage{listings} %For code blocks

%%%%Blackboard Bold%%%%%
\newcommand{\N}{{\mathbb N}}
\newcommand{\Z}{{\mathbb Z}}
\newcommand{\Q}{{\mathbb Q}}
\newcommand{\R}{{\mathbb R}}
\newcommand{\C}{{\mathbb C}}
\newcommand{\T}{{\mathbb T}}
\newcommand{\F}{{\mathbb F}}
\newcommand{\HH}{{\mathbb H}}

\newcommand{\compose}{\circ}
%%%%%Bold%%%%%%%%%
\newcommand{\bolda}{{\mathbf a}}
\newcommand{\boldb}{{\mathbf b}}
\newcommand{\boldc}{{\mathbf c}}
\newcommand{\boldd}{{\mathbf d}}
\newcommand{\bolde}{{\mathbf e}}
\newcommand{\boldi}{{\mathbf i}}
\newcommand{\boldj}{{\mathbf j}}
\newcommand{\boldk}{{\mathbf k}}
\newcommand{\boldn}{{\mathbf n}}
\newcommand{\boldp}{{\mathbf p}}
\newcommand{\boldq}{{\mathbf q}}
\newcommand{\boldr}{{\mathbf r}}
\newcommand{\bolds}{{\mathbf s}}
\newcommand{\boldt}{{\mathbf t}}
\newcommand{\boldu}{{\mathbf u}}
\newcommand{\boldv}{{\mathbf v}}
\newcommand{\boldw}{{\mathbf w}}
\newcommand{\boldx}{{\mathbf x}}
\newcommand{\boldy}{{\mathbf y}}
\newcommand{\boldz}{{\mathbf z}}
\newcommand{\boldzero}{{\mathbf 0}}
\newcommand{\boldmod}{\boldsymbol{ \bmod }}

\newcommand{\boldC}{{\mathbf C}}
\newcommand{\boldD}{{\mathbf D}}
\newcommand{\boldT}{{\mathbf T}}
\newcommand{\boldN}{{\mathbf N}}
\newcommand{\boldB}{{\mathbf B}}
\newcommand{\boldF}{{\mathbf F}}
\newcommand{\boldS}{{\mathbf S}}
\newcommand{\boldG}{{\mathbf G}}
\newcommand{\boldK}{{\mathbf K}}
\newcommand{\boldL}{{\mathbf L}}
\newcommand{\boldX}{{\mathbf X}}
\newcommand{\boldY}{{\mathbf Y}}
\newcommand{\boldZ}{{\mathbf Z}}
\newcommand{\boldH}{{\mathbf H}}
\newcommand{\boldR}{{\mathbf R}}

%%%%%%%%%%%OPERATORS%%%%%%%%%%%%%%%%%%
\renewcommand{\Re}{\operatorname{Re}}
\renewcommand{\Im}{\operatorname{Im}}

\DeclareMathOperator{\lcm}{lcm}
\DeclareMathOperator{\Span}{span}
\DeclareMathOperator{\tr}{tr}
\DeclareMathOperator{\NS}{null}
\DeclareMathOperator{\RS}{row}
\DeclareMathOperator{\CS}{col}
\DeclareMathOperator{\im}{im}
\DeclareMathOperator{\range}{range}
\DeclareMathOperator{\rank}{rank}
\DeclareMathOperator{\nullity}{nullity}
\DeclareMathOperator{\sign}{sign}
\DeclareMathOperator{\Fix}{Fix}
\DeclareMathOperator{\Aff}{Aff}
\DeclareMathOperator{\Frac}{Frac}
\DeclareMathOperator{\Ann}{Ann}
\DeclareMathOperator{\Tor}{Tor}
\DeclareMathOperator{\id}{id}
\DeclareMathOperator{\mdeg}{mdeg}
\DeclareMathOperator{\Lt}{Lt}
\DeclareMathOperator{\Lc}{Lc}
\DeclareMathOperator{\disc}{disc}
\DeclareMathOperator{\Frob}{Frob}
\DeclareMathOperator{\adj}{adj}
%\DeclareMathOperator{\proj}{proj}
\DeclareMathOperator{\curl}{curl}
\DeclareMathOperator{\grad}{grad}
\DeclareMathOperator{\diver}{div}
\DeclareMathOperator{\flux}{flux}
\DeclareMathOperator{\cis}{cis}
\DeclareMathOperator{\Arg}{Arg}
\DeclareMathOperator{\Log}{Log}
\DeclareMathOperator{\Arcsin}{Arcsin}
\DeclareMathOperator{\Arccos}{Arccos}
\DeclareMathOperator{\Arctan}{Arctan}
\DeclareMathOperator{\Res}{Res}
\DeclareMathOperator{\Int}{Int}
\DeclareMathOperator{\Ext}{Ext}
\DeclareMathOperator{\Isom}{Isom}
\DeclareMathOperator{\Nm}{Nm}
\DeclareMathOperator{\irr}{irr}
\def\Gal{\operatorname{Gal}}
\def\ord{\operatorname{ord}}
\def \ML {\operatorname{M}}
\def \GL {\operatorname{GL}}
\def \PGL {\operatorname{PGL}}
\def \SL {\operatorname{SL}}
\def \PSL {\operatorname{PSL}}
\def \GSp {\operatorname{GSp}}
\def \PGSp {\operatorname{PGSp}}
\def \Sp {\operatorname{Sp}}
\def \PSp {\operatorname{PSp}}
\def\Aut{\operatorname{Aut}}
\def\Inn{\operatorname{Inn}}
\def\Hom{\operatorname{Hom}}
\def\End{\operatorname{End}}
\def\ch{\operatorname{char}}


%%%%%%%Shortcuts and new commands %%%%%%%%
\def\Zp{\Z/p\Z}
\def\Zm{\Z/m\Z}
\def\Zn{\Z/n\Z}
\def\Fp{\F_p}

\newcommand{\surjects}{\twoheadrightarrow}
\newcommand{\injects}{\hookrightarrow}
\newcommand{\bijects}{\leftrightarrow}
\newcommand{\isomto}{\overset{\sim}{\rightarrow}}
\newcommand{\floor}[1]{\lfloor#1\rfloor}
\newcommand{\ceiling}[1]{\left\lceil#1\right\rceil}
\newcommand{\mclass}[2][m]{[#2]_{#1}}
\newcommand{\val}[2][]{\left\lvert #2\right\rvert_{#1}}
\newcommand{\abs}[2][]{\left\lvert #2\right\rvert_{#1}}
\newcommand{\valuation}[2][]{\left\lvert #2\right\rvert_{#1}}
\newcommand{\norm}[1]{\left\lVert#1\right\rVert}
\newcommand{\anpoly}{a_nx^n+a_{n-1}x^{n-1}\cdots +a_1x+a_0}
\newcommand{\anmonic}{x^n+a_{n-1}x^{n-1}\cdots +a_1x+a_0}
\newcommand{\bmpoly}{b_mx^m+b_{m-1}x^{m-1}\cdots +b_1x+b_0}
\newcommand{\pder}[2]{\frac{\partial#1}{\partial#2}}
\renewcommand{\c}{\cancel}
\newcommand{\normalin}{\trianglelefteq}
\newcommand{\angvec}[1]{\langle #1\rangle}
\newcommand{\varpoly}[2]{#1_{#2}x^{#2}+#1_{#2-1}x^{#2-1}\cdots +#1_1x+#1_0}
\newcommand{\varpower}[1][a]{#1_0+#1_1x+#1_1x^2+\cdots}
\newcommand{\limasto}[2][x]{\lim_{#1\rightarrow #2}}
\newcommand{\proj}[2]{\mbox{proj}_{#2}({#1}) }
\newcommand{\notimplies}{
        \mathrel{{\ooalign{\hidewidth$\not\phantom{=}$\hidewidth\cr$\implies$}}}}
\def\ntoinfty{\lim_{n\rightarrow\infty}}
\def\xtoinfty{\lim_{x\rightarrow\infty}}

\def\ii{\item}
\def\bb{\begin{enumerate}}
\def\ee{\end{enumerate}}
\def\ds{\displaystyle}
\def\p{\partial}

%\newenvironment{linsys}[2][m]{%
%\setlength{\arraycolsep}{.1111em} % p. 170 TeXbook; a medmuskip
%\begin{array}[#1]{@{}*{#2}{rc}r@{}}
%}{%
%\end{array}}

%\newenvironment{solution}{\begin{proof}[Solution]}{\end{proof}}

%%%%%%%%FANCY HEADER%%%%%%%%%
\pagestyle{plain}
\setlength{\headheight}{13.6pt}
\fancyhfoffset[L]{.5in}
%\lhead{\Large \bf{Name:}}
\chead{Executive summary: inverse trigonometric functions}
\rhead{Math 220-2}
%\lfoot{TURN OVER!}
%\rfoot{TURN OVER!}

%%%%%%%PAGE LAYOUT%%%%%%%%%%%%%
\setlength{\textwidth}{6.5in}
\setlength{\textheight}{9in}

%\setlength{\topmargin}{-.8in}
%\setlength{\columnsep}{1.5in}
\addtolength{\hoffset}{-1 in}
\addtolength{\voffset}{-.5 in}


%%%%%%%THEOREM ENVIRONMENTS%%%%%%%%
\theoremstyle{definition}
\newtheorem*{definition}{Definition}
\newtheorem*{definitions}{Definitions}
\newtheorem*{notation}{Notation}
\newtheorem*{example}{Example}
\newtheorem*{comment}{Comment}
\newtheorem*{comments}{Comments}
\newtheorem*{examples}{Examples}
\newtheorem*{warning}{Warning}
\newtheorem*{theorem}{Theorem}
\newtheorem*{corollary}{Corollary}
\newtheorem*{proposition}{Proposition}
\newtheorem*{lemma}{Lemma}

\newtheoremstyle{named}{}{}{}{}{\bfseries}{.}{.5em}{\thmnote{#3}}
\theoremstyle{named}
\newtheorem*{namedtheorem}{Theorem}

\newcounter{myalgctr}
\newenvironment{myalg}{%      define a custom environment
   \bigskip\noindent%         create a vertical offset to previous material
   \refstepcounter{myalgctr}% increment the environment's counter
   \textbf{Algorithm \themyalgctr}% or \textbf, \textit, ...
   \newline%
   }{\par\bigskip}  %
\numberwithin{myalgctr}{section}



\newenvironment{solution}{\begin{proof}[Solution]}{\end{proof}}


%%%%%%%%%%HYPERREFS PACKAGE%%%%%%%%%%%%%%%%%
\usepackage[colorlinks]{hyperref}
%\definecolor{webcolor}{rgb}{0.8,0,0.2}
%\definecolor{webbrown}{rgb}{.6,0,0}
%\usepackage[
%        colorlinks,
%       linkcolor=webbrown,  filecolor=webcolor,  citecolor=webbrown,
%        backref
%]{hyperref}
\usepackage[alphabetic, lite]{amsrefs} % for bibliography
\begin{document}
\thispagestyle{fancy}
\subsection*{Definitions}
\begin{namedtheorem}[Inverse trigonometric functions] The following are examples of what are called {\bf inverse trigonometric functions}.
  \begin{itemize}
    \item On the restricted domain $[-\pi/2, \pi/2]$ the function $f(x)=\sin x$ is one-to-one, with range $[-1,1]$. The inverse function of $f$ restricted to this domain is called the {\bf arcsine function}, denoted $f^{-1}(x)=\arcsin x$.
    \item On the restricted domain $[0, \pi]$ the function $g(x)=\cos x$ is one-to-one, with range $[-1,1]$. The inverse function of $g$ restricted to this domain is called the {\bf arccosine function}, denoted $g^{-1}(x)=\arccos x$.
    \item On the restricted domain $(-\pi/2, \pi/2)$ the function $h(x)=\tan x$ is one-to-one, with range $(-\infty, \infty)$. The inverse function of $h$ restricted to this domain is called the {\bf arctangent function}, denoted $h^{-1}(x)=\arctan x$.
  \end{itemize}

\end{namedtheorem}
\begin{comment}
Occasionally an alternative notation is used to denote inverse trig functions: namely,
\begin{align*}
  \arcsin x&=\sin^{-1} x & \arccos x&=\cos^{-1} x & \arctan x&=\tan^{-1} x.
\end{align*}
We will avoid this alternative notation as it misleadingly suggests these inverse trigonometric functions are {\em reciprocals} of the corresponding trigonometric functions. They are not!
\end{comment}

%***********************************************
 \subsection*{Theory}

\begin{namedtheorem}[Properties of inverse trigonometric functions]\
  \begin{itemize}
    \item The function $\arcsin$ is an increasing function with domain $[-1,1]$ and range $[0,\pi]$. It satisfies the following properties:
    \begin{align*}
      \arcsin(x)=\theta &\iff \sin\theta=x \text{ and } -\pi/2\leq \theta\leq \pi/2\\
      \arcsin(\sin \theta)&=\theta \text{ for all } -\pi/2\leq \theta\leq \pi/2\\
      \sin(\arcsin x)&=x \text{ for all } -1\leq x\leq 1.
    \end{align*}
    \item The function $\arccos$ is a decreasing function with domain $[-1,1]$ and range $[0,\pi]$. It satisfies the following properties:
    \begin{align*}
      \arccos(x)=\theta &\iff \cos\theta=x \text{ and } 0\leq \theta\leq \pi\\
      \arccos(\cos \theta)&=\theta \text{ for all } 0\leq \theta\leq \pi\\
      \cos(\arccos x)&=x \text{ for all } -1\leq x\leq 1.
    \end{align*}
    \item The function $\arctan$ is an increasing function with domain $(-\infty, \infty)$ and range $(-\pi/2, \pi/2)$. It satisfies the following properties:
    \begin{align*}
      \arctan(x)=\theta &\iff \tan\theta=x \text{ and } -\pi/2< \theta< \pi/2\\
      \arctan(\tan \theta)&=\theta \text{ for all } -\pi/2< \theta< \pi/2\\
      \tan(\arctan x)&=x \text{ for all } x\\
      \lim_{x\to\infty}\arctan x&=\pi/2,\hspace{5pt} \lim_{x\to-\infty}\arctan x=-\pi/2
    \end{align*}
  \end{itemize}

\end{namedtheorem}
\begin{namedtheorem}[Derivative formulas for inverse trigonometric functions] The following derivative/antiderivative formulas hold:
  \begin{align*}
    \frac{d}{dx} \arcsin x=\frac{1}{\sqrt{1-x^2}}&\iff \int \frac{1}{\sqrt{1-x^2}}\, dx=\arcsin x+C &\text{(for all $x$ in $(-1,1)$)}\\
    \frac{d}{dx} \arccos x=-\frac{1}{\sqrt{1-x^2}}&\iff \int \frac{1}{\sqrt{1-x^2}}\, dx=-\arccos x+C &\text{(for all $x$ in $(-1,1)$)}\\
    \frac{d}{dx} \arctan x=\frac{1}{1+x^2}&\iff \int \frac{1}{1+x^2}\, dx=\arctan x+C &\text{(for all $x$)}.
  \end{align*}

\end{namedtheorem}
%***************************************

% \subsection*{Procedures}

%*********************************************************
\subsection*{Examples}

\begin{enumerate}
  \item Compute the following values of trigonometric functions by hand.
  \begin{enumerate}
    \item $\displaystyle\arcsin(-1)$
    \item $\displaystyle\arccos(-\sqrt{2}/2)$
    \item $\displaystyle\arctan(-1/\sqrt{3})$
    \item $\displaystyle\arcsin\left(\sin\left(\frac{10\pi}{11}\right)\right)$

    {\bf Hint}. The answer is not $10\pi/11$.
  \end{enumerate}
  \item Find all solutions to the following trigonometric equations lying within the interval $[0,2\pi]$. You may express your answer in terms of values of inverse trigonometric functions.
  \begin{enumerate}
    \item $\displaystyle 3\sin 2\theta +4=6$
    \item $\displaystyle \tan (\theta+\pi)=-10$
  \end{enumerate}
  \item Find the equation of the tangent line to $f(x)=\arccos x $ at $x=1/2$.

  \item Compute $\displaystyle\lim_{x\rightarrow 1^{-}}\frac{\arccos(x^2)}{\sqrt{1-x}}$

  \item Compute $\displaystyle\int \frac{x+1}{\sqrt{1-(x+2)^2}}\, dx$.
\end{enumerate}



\end{document}
