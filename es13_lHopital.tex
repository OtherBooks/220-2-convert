\documentclass[11pt]{article}
%%%%%%%%%%PACKAGES%%%%%%%%%%%%%%%%%%%%%%%%%%%%%%%%%%%
\usepackage{latexsym}
\usepackage{amssymb, amsmath, amsthm, amsfonts}
\usepackage{stmaryrd} %For \mapsfrom
%\usepackage[fleqn]{amsmath}  % fleqn option makes aligned equations flushed left!
%\usepackage[english]{babel}
%\usepackage{pgf}
\usepackage{mathtools}
\usepackage[mathscr]{eucal}
\usepackage{fancyhdr}
\usepackage{multicol,parcolumns}
\usepackage{enumerate}
%\usepackage{enumitem}
\usepackage[shortlabels]{enumitem}
\usepackage{graphicx}
\usepackage{extarrows}
\usepackage{cancel}
%\usepackage{tikz}
%\usepackage[all,cmtip]{xy} %\SelectTips{cm}{10}
\usepackage[all]{xy} \SelectTips{cm}{10}
%\usepackage{listings} %For code blocks

%%%%Blackboard Bold%%%%%
\newcommand{\N}{{\mathbb N}}
\newcommand{\Z}{{\mathbb Z}}
\newcommand{\Q}{{\mathbb Q}}
\newcommand{\R}{{\mathbb R}}
\newcommand{\C}{{\mathbb C}}
\newcommand{\T}{{\mathbb T}}
\newcommand{\F}{{\mathbb F}}
\newcommand{\HH}{{\mathbb H}}

\newcommand{\compose}{\circ}
%%%%%Bold%%%%%%%%%
\newcommand{\bolda}{{\mathbf a}}
\newcommand{\boldb}{{\mathbf b}}
\newcommand{\boldc}{{\mathbf c}}
\newcommand{\boldd}{{\mathbf d}}
\newcommand{\bolde}{{\mathbf e}}
\newcommand{\boldi}{{\mathbf i}}
\newcommand{\boldj}{{\mathbf j}}
\newcommand{\boldk}{{\mathbf k}}
\newcommand{\boldn}{{\mathbf n}}
\newcommand{\boldp}{{\mathbf p}}
\newcommand{\boldq}{{\mathbf q}}
\newcommand{\boldr}{{\mathbf r}}
\newcommand{\bolds}{{\mathbf s}}
\newcommand{\boldt}{{\mathbf t}}
\newcommand{\boldu}{{\mathbf u}}
\newcommand{\boldv}{{\mathbf v}}
\newcommand{\boldw}{{\mathbf w}}
\newcommand{\boldx}{{\mathbf x}}
\newcommand{\boldy}{{\mathbf y}}
\newcommand{\boldz}{{\mathbf z}}
\newcommand{\boldzero}{{\mathbf 0}}
\newcommand{\boldmod}{\boldsymbol{ \bmod }}

\newcommand{\boldC}{{\mathbf C}}
\newcommand{\boldD}{{\mathbf D}}
\newcommand{\boldT}{{\mathbf T}}
\newcommand{\boldN}{{\mathbf N}}
\newcommand{\boldB}{{\mathbf B}}
\newcommand{\boldF}{{\mathbf F}}
\newcommand{\boldS}{{\mathbf S}}
\newcommand{\boldG}{{\mathbf G}}
\newcommand{\boldK}{{\mathbf K}}
\newcommand{\boldL}{{\mathbf L}}
\newcommand{\boldX}{{\mathbf X}}
\newcommand{\boldY}{{\mathbf Y}}
\newcommand{\boldZ}{{\mathbf Z}}
\newcommand{\boldH}{{\mathbf H}}
\newcommand{\boldR}{{\mathbf R}}

%%%%%%%%%%%OPERATORS%%%%%%%%%%%%%%%%%%
\renewcommand{\Re}{\operatorname{Re}}
\renewcommand{\Im}{\operatorname{Im}}

\DeclareMathOperator{\lcm}{lcm}
\DeclareMathOperator{\Span}{span}
\DeclareMathOperator{\tr}{tr}
\DeclareMathOperator{\NS}{null}
\DeclareMathOperator{\RS}{row}
\DeclareMathOperator{\CS}{col}
\DeclareMathOperator{\im}{im}
\DeclareMathOperator{\range}{range}
\DeclareMathOperator{\rank}{rank}
\DeclareMathOperator{\nullity}{nullity}
\DeclareMathOperator{\sign}{sign}
\DeclareMathOperator{\Fix}{Fix}
\DeclareMathOperator{\Aff}{Aff}
\DeclareMathOperator{\Frac}{Frac}
\DeclareMathOperator{\Ann}{Ann}
\DeclareMathOperator{\Tor}{Tor}
\DeclareMathOperator{\id}{id}
\DeclareMathOperator{\mdeg}{mdeg}
\DeclareMathOperator{\Lt}{Lt}
\DeclareMathOperator{\Lc}{Lc}
\DeclareMathOperator{\disc}{disc}
\DeclareMathOperator{\Frob}{Frob}
\DeclareMathOperator{\adj}{adj}
%\DeclareMathOperator{\proj}{proj}
\DeclareMathOperator{\curl}{curl}
\DeclareMathOperator{\grad}{grad}
\DeclareMathOperator{\diver}{div}
\DeclareMathOperator{\flux}{flux}
\DeclareMathOperator{\cis}{cis}
\DeclareMathOperator{\Arg}{Arg}
\DeclareMathOperator{\Log}{Log}
\DeclareMathOperator{\Arcsin}{Arcsin}
\DeclareMathOperator{\Arccos}{Arccos}
\DeclareMathOperator{\Arctan}{Arctan}
\DeclareMathOperator{\Res}{Res}
\DeclareMathOperator{\Int}{Int}
\DeclareMathOperator{\Ext}{Ext}
\DeclareMathOperator{\Isom}{Isom}
\DeclareMathOperator{\Nm}{Nm}
\DeclareMathOperator{\irr}{irr}
\def\Gal{\operatorname{Gal}}
\def\ord{\operatorname{ord}}
\def \ML {\operatorname{M}}
\def \GL {\operatorname{GL}}
\def \PGL {\operatorname{PGL}}
\def \SL {\operatorname{SL}}
\def \PSL {\operatorname{PSL}}
\def \GSp {\operatorname{GSp}}
\def \PGSp {\operatorname{PGSp}}
\def \Sp {\operatorname{Sp}}
\def \PSp {\operatorname{PSp}}
\def\Aut{\operatorname{Aut}}
\def\Inn{\operatorname{Inn}}
\def\Hom{\operatorname{Hom}}
\def\End{\operatorname{End}}
\def\ch{\operatorname{char}}


%%%%%%%Shortcuts and new commands %%%%%%%%
\def\Zp{\Z/p\Z}
\def\Zm{\Z/m\Z}
\def\Zn{\Z/n\Z}
\def\Fp{\F_p}

\newcommand{\surjects}{\twoheadrightarrow}
\newcommand{\injects}{\hookrightarrow}
\newcommand{\bijects}{\leftrightarrow}
\newcommand{\isomto}{\overset{\sim}{\rightarrow}}
\newcommand{\floor}[1]{\lfloor#1\rfloor}
\newcommand{\ceiling}[1]{\left\lceil#1\right\rceil}
\newcommand{\mclass}[2][m]{[#2]_{#1}}
\newcommand{\val}[2][]{\left\lvert #2\right\rvert_{#1}}
\newcommand{\abs}[2][]{\left\lvert #2\right\rvert_{#1}}
\newcommand{\valuation}[2][]{\left\lvert #2\right\rvert_{#1}}
\newcommand{\norm}[1]{\left\lVert#1\right\rVert}
\newcommand{\anpoly}{a_nx^n+a_{n-1}x^{n-1}\cdots +a_1x+a_0}
\newcommand{\anmonic}{x^n+a_{n-1}x^{n-1}\cdots +a_1x+a_0}
\newcommand{\bmpoly}{b_mx^m+b_{m-1}x^{m-1}\cdots +b_1x+b_0}
\newcommand{\pder}[2]{\frac{\partial#1}{\partial#2}}
\renewcommand{\c}{\cancel}
\newcommand{\normalin}{\trianglelefteq}
\newcommand{\angvec}[1]{\langle #1\rangle}
\newcommand{\varpoly}[2]{#1_{#2}x^{#2}+#1_{#2-1}x^{#2-1}\cdots +#1_1x+#1_0}
\newcommand{\varpower}[1][a]{#1_0+#1_1x+#1_1x^2+\cdots}
\newcommand{\limasto}[2][x]{\lim_{#1\rightarrow #2}}
\newcommand{\proj}[2]{\mbox{proj}_{#2}({#1}) }
\newcommand{\notimplies}{
        \mathrel{{\ooalign{\hidewidth$\not\phantom{=}$\hidewidth\cr$\implies$}}}}
\def\ntoinfty{\lim_{n\rightarrow\infty}}
\def\xtoinfty{\lim_{x\rightarrow\infty}}

\def\ii{\item}
\def\bb{\begin{enumerate}}
\def\ee{\end{enumerate}}
\def\ds{\displaystyle}
\def\p{\partial}

%\newenvironment{linsys}[2][m]{%
%\setlength{\arraycolsep}{.1111em} % p. 170 TeXbook; a medmuskip
%\begin{array}[#1]{@{}*{#2}{rc}r@{}}
%}{%
%\end{array}}

%\newenvironment{solution}{\begin{proof}[Solution]}{\end{proof}}

%%%%%%%%FANCY HEADER%%%%%%%%%
\pagestyle{plain}
\setlength{\headheight}{13.6pt}
\fancyhfoffset[L]{.5in}
%\lhead{\Large \bf{Name:}}
\chead{Executive summary: l'H\^opital's rule}
\rhead{Math 220-2}
%\lfoot{TURN OVER!}
%\rfoot{TURN OVER!}

%%%%%%%PAGE LAYOUT%%%%%%%%%%%%%
\setlength{\textwidth}{6.5in}
\setlength{\textheight}{9in}

%\setlength{\topmargin}{-.8in}
%\setlength{\columnsep}{1.5in}
\addtolength{\hoffset}{-1 in}
\addtolength{\voffset}{-.5 in}


%%%%%%%THEOREM ENVIRONMENTS%%%%%%%%
\theoremstyle{definition}
\newtheorem*{definition}{Definition}
\newtheorem*{definitions}{Definitions}
\newtheorem*{notation}{Notation}
\newtheorem*{example}{Example}
\newtheorem*{comment}{Comment}
\newtheorem*{comments}{Comments}
\newtheorem*{examples}{Examples}
\newtheorem*{warning}{Warning}
\newtheorem*{theorem}{Theorem}
\newtheorem*{corollary}{Corollary}
\newtheorem*{proposition}{Proposition}
\newtheorem*{lemma}{Lemma}

\newtheoremstyle{named}{}{}{}{}{\bfseries}{.}{.5em}{\thmnote{#3}}
\theoremstyle{named}
\newtheorem*{namedtheorem}{Theorem}

\newcounter{myalgctr}
\newenvironment{myalg}{%      define a custom environment
\bigskip\noindent%         create a vertical offset to previous material
\refstepcounter{myalgctr}% increment the environment's counter
\textbf{Algorithm \themyalgctr}% or \textbf, \textit, ...
\newline%
}{\par\bigskip}  %
\numberwithin{myalgctr}{section}



\newenvironment{solution}{\begin{proof}[Solution]}{\end{proof}}


%%%%%%%%%%HYPERREFS PACKAGE%%%%%%%%%%%%%%%%%
\usepackage[colorlinks]{hyperref}
%\definecolor{webcolor}{rgb}{0.8,0,0.2}
%\definecolor{webbrown}{rgb}{.6,0,0}
%\usepackage[
%        colorlinks,
%       linkcolor=webbrown,  filecolor=webcolor,  citecolor=webbrown,
%        backref
%]{hyperref}
\usepackage[alphabetic, lite]{amsrefs} % for bibliography
\begin{document}
\thispagestyle{fancy}
\subsection*{Definitions}
\begin{namedtheorem}[Indeterminate forms]Consider a limit expression of the form
  \[
  \lim_{x\to a}\frac{f(x)}{g(x)},
  \]
  where $a$ is either a finite number or $\pm\infty$.
  \vspace{.1in}
  \\
  The expression is an {\bf indeterminate form of type $0/0$} if
  \[
  \lim_{x\to a}f(x)=\lim_{x\to a}g(x)=0.
  \]
  The expression is an {\bf indeterminate form of type $\infty/\infty$} if
  \[
  \lim_{x\to a}f(x)=\pm\infty \text{ and } \lim_{x\to a}g(x)=\pm\infty.
  \]

\end{namedtheorem}
\begin{comment}
  A limit expression having an indeterminate form does {\em not} mean that the limit does not exist. You should interpret this conclusion as saying simply that our current analysis is not detailed enough to determine whether the limit exists and/or what that limit is.
  \vspace{.1in}
  \\
  In this spirit, we will be careful {\em not} to write
  \[
  \lim_{x\to a}\frac{f(x)}{g(x)}=\frac{0}{0} \text{ or } \lim_{x\to a}\frac{f(x)}{g(x)}=\frac{\infty}{\infty},
  \]
  as this suggests we are asserting something more definitive about the limit.

\end{comment}
\begin{namedtheorem}[Further indeterminate forms] Assume $a$ is either a finite number or $\pm\infty$.
  \vspace{.1in}
  \\
  If $\displaystyle\lim_{x\to a}f(x)=\lim_{x\to a}g(x)=\infty$, then
  $\displaystyle\lim_{x\to a}f(x)-g(x)$ is an {\bf indeterminate form of type $\infty-\infty$}.
  \vspace{.1in}
  \\
  If $\displaystyle\lim_{x\to a}f(x)=0$ and $\lim_{x\to a}g(x)=\pm\infty$, then
  $\displaystyle\lim_{x\to a}f(x)g(x)$ is an {\bf indeterminate \\ form of type $0\cdot\infty$}.
  \vspace{.1in}
  \\
  If $\displaystyle\lim_{x\to a}f(x)=\lim_{x\to a}g(x)=0$, then
  $\displaystyle\lim_{x\to a}f(x)^{g(x)}$ is an {\bf indeterminate form of type $0^0$}.
  \vspace{.1in}
  \\
  If $\displaystyle\lim_{x\to a}f(x)=\infty$ and $\lim_{x\to a}g(x)=0$, then
  $\displaystyle\lim_{x\to a}f(x)^{g(x)}$ is an {\bf indeterminate form  of type $\infty^0$}.
  \vspace{.1in}
  \\
  If $\displaystyle\lim_{x\to a}f(x)=1$ and $\lim_{x\to a}g(x)=\infty$, then
  $\displaystyle\lim_{x\to a}f(x)^{g(x)}$ is an {\bf indeterminate form of type $1^\infty$}.

\end{namedtheorem}

%***********************************************
\subsection*{Theory}
\begin{namedtheorem}[L'H\^opital's rule] Let $f$ and $g$ be differentiable on an open interval $I$ containing $a$, where $a$ is either a finite number or $\pm\infty$, and suppose $g'(x)\ne 0$ for all $x\ne a$ in the interval.
  \vspace{.1in}
  \\
  If $\displaystyle\lim_{x\to a}\frac{f(x)}{g(x)}$ is an indeterminate form of type $0/0$ or $\infty/\infty$, then
  \[
  \lim_{x\to a}\frac{f(x)}{g(x)}=\lim_{x\to a}\frac{f'(x)}{g'(x)},
  \]
  provided the limit on the right exists or is equal to $\pm\infty$. \\
  The same result holds if we replace the limit with a one-sided limit.

\end{namedtheorem}
\begin{comment}
Students tend to fall madly in love with l'H\^opital's rule after seeing it for the first time. Some comments to temper your passion:
\begin{enumerate}
  \item Make sure the necessary conditions hold: (a) $f,g$ differentiable on an interval about $a$, $g(x)\ne 0$ on for $x\ne a$, and the limit expression is indeterminate of type $0/0$ or $\infty/\infty$.
  \item As magic as the rule appears, there are many examples where either the application of this rule does not help, and/or it is easier to use a different technique. Consider the following limits, for example:
  \[
  \lim_{x\to\infty}\frac{e^x+e^{-x}}{e^x-e^{-x}}, \hspace{10pt} \lim_{x\rightarrow \infty}\frac{x^4-x^2+5x+7}{2x^4+x^3+x^2+x+1}
  \]
\end{enumerate}
\end{comment}

%***************************************

\subsection*{Procedures}

%*********************************************************
\subsection*{Examples}
\begin{enumerate}
  \item Decide whether the following limit expressions have determinate or indeterminate forms. If determinate, compute the limit.
  \begin{enumerate}
    \item $\displaystyle\lim_{x\to 0+}\frac{\sin x}{\ln x}$
    \item $\displaystyle\lim_{x\to (\pi/2)^-}\frac{\tan x}{\cos x}$
    \item $\displaystyle\lim_{x\to\infty}\frac{e^x}{2^x+3^x}$
  \end{enumerate}
  \item Compute the following limits.
  \begin{enumerate}
    \item $\displaystyle\lim_{x\to\infty}\frac{\ln x}{x^{1000}}$
    \item $\displaystyle\lim_{x\to 0}\frac{2^x-3^{-x}}{4^x-5^{-x}}$
    \item $\displaystyle\lim_{x\to 1}\frac{\cos(\pi x/2)}{\log_2(x)}$
    \item $\displaystyle\lim_{x\to 0}\frac{x-\sin x}{x\sin x}$
  \end{enumerate}
  \item Compute the following limits.
  \begin{enumerate}
    \item $\displaystyle\lim_{x\to 0^+}\frac{1}{\sin x}-\frac{1}{x}$
    \item $\displaystyle\lim_{x\to \infty}2x-\sqrt{4x^2-13x}$
    \item $\displaystyle\lim_{x\to -\infty}x^22^{x}$
    \item $\displaystyle\lim_{x\to 0^+}(1+x)^{1/x}$
    \item $\displaystyle\lim_{x\to \infty}(1+x^2)^{2/x}$
  \end{enumerate}
\end{enumerate}




\end{document}
